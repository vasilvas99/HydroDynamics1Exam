\section{Равнинни стационарни течения на идеален несвиваем флуид. Комплексен потенциал и комплексна скорост.}
%%Уравнения на движение на равнинни стационарни безвихрови течения на идеален несвиваем флуид. Потенциал на скоростите Функция на тока. Комплексен потенциал и комплексна скорост. Примери.%%
\subsection{Равнинни стационарни течения на идеален несвиваем флуид.}
\subsubsection{Поставновка на задачата}
За потенциалните течения, можем да представим полето на скоростите като градиент на потенциал
\begin{equation*}
    \vec{v} = \nabla \Phi
\end{equation*}
Където $\Phi$ е някоя скаларна функция. Причината за въвеждане на такъв потенциал е, че такива течения се анализират относително по-лесно. Замествайки горното в условието за несвиваем флуид ($\nabla \cdot \vec{v} = 0$), получаваме че:

\noindent При потенциалните, несвиваеми флуиди, потенциалът $\Phi$ изпълнява уравнението на Лаплас:
\begin{equation*}
	\nabla^2 \Phi  = 0
\end{equation*}
За случая на стационарно равнинно, течение можем да запишем:
\begin{align}
    \label{eq:planar_flow}
	\frac{\partial u}{\partial x} + \frac{\partial v}{\partial y} = 0 \\
	u = \frac{\partial \Phi}{\partial x}, v = \frac{\partial \Phi}{\partial y}
\end{align}
Където за равнинно, стационарно, течение: $\Phi = \Phi(x,y)$
Тогава прилагайки правилото за приозводна на сложна функция, можем да получим за пълния диференциал на потенциала:
\begin{equation*}
	d \Phi = \frac{\partial \Phi}{\partial x} d x +  \frac{\partial \Phi}{\partial y} dy = u dx + v dy
\end{equation*}
За да затворим диференциалната задача, трябва да наложим съответните гранични условия.
\begin{align*}
	\frac{\partial \Phi}{\partial x} \Bigg|_{\infty} & = U_{\infty} \\
	\frac{\partial \Phi}{\partial y} \Bigg|_{\infty} & = V_{\infty} \\
	\frac{\partial \Phi}{\partial \vec{n}} \Bigg|_{S}      & = 0
\end{align*}
Където $S$ e повърхнината на тялото, а $\vec{n}$ - нормалата към тази повърхнина. Това последно г.у. е условие за непроникване на 
флуида в тялото.
Уравнението на Лаплас и горните г.у. водят до задача на Нойман с хомогенно г.у. върху повърхността.
\subsubsection{Функция на тока}
Въвеждаме функцията $\Psi(x,y)$, такава че:
\begin{align}
    \label{eq:flow_func_uv}
    u &= \frac{\partial \Psi}{\partial y} \\
    v &= -\frac{\partial \Psi}{\partial x} \nonumber
\end{align}
Така \autoref{eq:leibniz} бива автоматично удовлетворено. $\Psi(x,y)$ наричаме функция на тока, т.като изолиниите $\left(\{(x,y): \Psi(x,y) = const\}\right)$ съвпадат с линиите на тока на даденото течение.
Горното лесно се проверява чрез определението за линиите на тока:
\begin{equation*}
    \frac{dx}{u(x,y)} = \frac{dy}{v(x,y)} 
\end{equation*}
Заместваме \autoref{eq:flow_func_uv} в горното и получаваме:
\begin{equation*}
    \frac{\partial \Psi}{\partial x} d x + \frac{\partial \Psi}{\partial y} d y = 0
\end{equation*}
С което доказахме твърдението, че изолиниите на функцията на тока са именно линиите на тока.

Функцията на тока можем да интерпретираме лесно, като за целта нека пресметнем количеството флуид, протичащ през кривата AB в дадена равнината. От дефиницията на потока, можем лесно да запишем интегралното количество флуид:
\begin{equation*}
    Q = \int_{AB} = \vec{v} \cdot \vec{n} ds
\end{equation*}
Където е $\vec{n}$ е нормалата към кривата AB. Лесно се проверява, чe $\vec{n} = (\frac{dy}{ds}, -\frac{d x}{d s})$. Прилагайки теоремата на Грийн, съвсем директно получаваме:
\begin{equation}
    Q = \int_{AB} = \vec{v}.vec{n} ds = \int_{AB} \nabla \Psi \cdot \vec{dr} = \int_{A}^{B} d \Psi = \Psi(B) - \Psi(A) 
\end{equation}
Това означава, че количеството течност, което се движи между две линии на тока е равно на разликата от стойностите на функцията на тока върху тези линии (консервативно поле).
Освен това, лесно се вижда, че:
\begin{align*}
    \Omega_z = (\nabla \times \vec{v}) \cdot \vec{k} = - \nabla^2 \Psi
\end{align*}
Тогава ако имаме безвихрово равнинно течение, горното преминава в уравнение на Лаплас за $\nabla^2 \Psi$.
Г.у. за функцията на тока на безкрайност са на Дирихле:
\begin{equation*}
    \nabla \Psi \big|_\infty = (-V_\infty, U_\infty)
\end{equation*}

Лесно се проверява, че нормалната компонента на скоростта на повърхността на тялото е $v_n = \vec{v} \cdot \vec{n} = \frac{\partial \Psi}{\partial s}$. За идеален флуид искаме $v_n =0$. За г.у. на тази граница получаваме:
\begin{align*}
    \frac{\partial \Psi}{\partial s}\big|_{S} &= 0 \\
    \therefore \Psi_S  &= const
\end{align*}
Крайно получихме задача на Дирихле за у-нието на Лаплас, която трябва да решим за да намерим функцията на тока. 

\subsubsection{Комплексен потенциал и комплексна скорост}
