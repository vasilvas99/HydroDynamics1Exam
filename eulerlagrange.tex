\section{Методи на Ойлер и Лагранж. Tеорема на Лайбниц.}
За да се опише кинематиката на абсолютно твърдите тела е достатъчно да е известно движението на три негови точки. Флуидите обаче, не могат да бъдат описани така. Това е защото
те са състваени от континиум от материални точки (,,частици``), които до голяма степен могат да се движат независимо една от друга.

Движението на един флуид е определено напълно, ако се познава скоростта $\vec{v}$ във всяка точка от флуида, т.е. знаем ,,как се движи`` всяка една от тези материални точки.
Тази скорост, разбира се, може да зависи от времето $t$. Т.е. в общия случай за движението на един флуид, разглеждан като континиум $\vec{v} = \vec{f}(t,\vec{r})$, където $t$ - време,
$\vec{r}$ - радиус-вектор на частицата.

В общия случай в 3D декартови координати $\vec{r} = x_1 \vec{i} + x_2 \vec{j} + x_3 \vec{k}$, където $\vec{i}, \vec{j}, \vec{k}$ - единичните базисни вектори на декартовата координатна система.
\subsection{Методи на Ойлер и Лагранж}
Двата основни подхода за моделиране на флуидна система при горните приближения са методите на Лагранж и Ойлер.
\subsubsection{Метод на Лагранж}
Този метод е директен аналог на методите за описание на кинематиката на абсолютно твърдо тяло - определя се траекторията на всяка от частиците като функция на времето. Особеността  тук, както беше споменато е,
че флуида се състои от ,,безкрайно много`` такива частици. За целта при флуидите трябва да се наложи н.у. за положението на всяка частица.

Нека за $t= t_0$, позицията е съответно: $\vec{r}(t_0) = (\xi^{0}_1,\xi^{0}_2,\xi^{0}_3)_{<\vec{i}, \vec{j}, \vec{k}>}$, тогава координатите на всяка частица $\vec{r}(t) = (x_1,x_2,x_3)_{<\vec{i}, \vec{j}, \vec{k}>}$ за 
всеки момент от времето е функция на началното положениe.
\begin{equation}
	\vec{r(t)} = \vec{f}(t, \vec{r}(t_0))
\end{equation}
Този подход ни дава лесен начин да намерим скоростта и съответните ѝ компоненти:
\begin{equation}
	\vec{v}(t, \vec{r}(t_0)) =  \frac{\partial \vec{r(t)}}{\partial t}
\end{equation}
\begin{align*}
	v_1 (t, \vec{r}(t_0)) & =  \frac{\partial f_1}{\partial t} \\
	v_2 (t, \vec{r}(t_0)) & =  \frac{\partial f_2}{\partial t} \\
	v_3 (t, \vec{r}(t_0)) & =  \frac{\partial f_3}{\partial t} 
\end{align*}
Аналогично, съвсем лесно при този подход можем да намерим компонентите на ускорението от траекториите като:
\begin{equation}
	a_{i} = \frac{\partial^2 x_i}{\partial t^2}
\end{equation}
Обобщено за този подход можем да кажем, че той е начин за моделиране на флуида като проследяваме движението на една частица. Изчертавайки позицията на тази частица като функция от времето, получаваме нейната траектория. Интуитивно за този подход
можем да си представяме, че се намираме в лодка и се носим по течението на една река. 

\subsubsection{Метод на Ойлер}
Този подход се фокусира върху харектеристиките на движението, което флуида извършва в даден момент от времето в опредлена геометрична точка. Т.е. вече не се интересуваме от историята на движението на всяка една частица от флуида, а се конкцентрираме
в една геометрична точка от обема му. През тази точка преминават различни флуидни частици с течението на времето. При това разглеждане, движението на флуида е напълно определено, ако сме скоростта са зададени като функции на координатите. Нека отбележим $\vec{\rho} = (x_1,x_2,x_3)_{<\vec{i}, \vec{j}, \vec{k}>}$,
където $x_i$ тук са координатите на разглежданата геометрична точка от обема. Тогава:
\begin{equation}
	\vec{v} = \frac{d \vec{\rho}}{d t} = \vec{F}(t, \vec{\rho})
\end{equation}
\begin{align*}
	\frac{d x_1}{d t} & = v_1 = F_1(t, x_1, x_2, x_3) \\
	\frac{d x_2}{d t} & = v_2 = F_2(t, x_1, x_2, x_3) \\
	\frac{d x_3}{d t} & = v_2 = F_3(t, x_1, x_2, x_3) 
\end{align*}

При този метод не се интересуваме, коя конкретна частица идва в дадената геометрична точка и как тя се движи по-нататък във времето. От математическа гледна точка, разликата тук е, че за променливи при подхода на Лагранж се разглеждат параметри описващи индивидуални частици от даден момент нататък, а при Ойлер - точките от пространството, което
заема флуида.
Интуитивно, методът на Ойлер е може да си представяме като да наблюдаваме фиксирана точка от реката от горния пример, стоейки на нейния бряг.

\subsubsection{Връзки между двата подхода}
Двата подхода целят да моделират една и съща физична система. Това означава, че 